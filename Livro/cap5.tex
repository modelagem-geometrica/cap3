\chapter{Variedades Definidas Implicitamente}\label{cap_var_imp}

\thispagestyle{empty}

Para maior aprofundamento no tema apresentado nesta e nas pr\'oximas se��es, \'e sugerida a leitura da tese~\cite{Cas92}, do livro~\cite{AlGe90}, do artigo~\cite{Cas06} e do curso no proceedings~\cite{Eaves76}. 

\markboth{Aproxima\c{c}\~oes de Fun\c{c}\~oes Impl\'{\i}citas}{Interpola\c{c}\~ao Linear Simplicial}
\section{Interpola\c{c}\~ao Linear Simplicial}\label{vi_ilp}


\begin{defi} $($Aproxima��o Simplicial$)$
Seja $F : S \subset  \R^{n} \rightarrow \R^{k}$ uma aplica\c c\~ao e $T$ uma triangula\c c\~ao de $S$.
Se $\sigma = [v_{0},\ldots,v_{n}] \in T$, tem-se para cada $v \in
\sigma$ uma \'unica $(n+1)$-upla $\lambda = (\lambda_{0}, \ldots,\lambda_{n})$ 
tal que $\sum_{i=0}^{n} \lambda_{i} v_{i} = v$, 
$\sum_{i=0}^{n} \lambda_{i} = 1$ e $\lambda_{i} \geq 0$, $i = 0,\ldots,n$.
Define-se $F_{\sigma} : \sigma \rightarrow \R^{k}$ onde
$F_{\sigma}(v) = F_{\sigma}(\sum_{i=0}^{n} \lambda_{i} v_{i}) = \sum_{i=0}^{n} \lambda_{i} F(v_{i})$.
\end{defi}

Observe que $F_{\sigma}$ \'e uma aplica\c c\~ao afim e que
$F_{\sigma}(v_{i}) = F(v_{i})$, $i = 0, \ldots, n$, isto \'e,
$F_{\sigma}$ \'e uma interpola\c c\~ao linear de $F$ nos v\'ertices de $\sigma$.

\begin{defi} $($Aproxima��o Linear por Partes$)$
Define-se uma aproxima\c c\~ao linear por partes para $F$ sendo, 
$F_{T} : S \subset  \R^{n} \rightarrow  \R^{k}$ onde 
$F_{T}(v) = F_{\sigma}(v)$ para $v \in \sigma \in T$.
\end{defi}

Portanto, $F_{T}$ \'e uma interpola\c c\~ao de $F$ para os
v\'ertices de $T$ (nos simplexos de dimens\~ao zero de $T$) e que \'e afim em
cada simplexo de dimens\~ao $n$ de $T$.

\begin{teore} \label{teo_AG}
Sejam $S \subset \R^{n}$ aberto,  $F : S \rightarrow \R^{k}$  uma
aplica\c c\~ao de classe $C^{2}$ com $\parallel D^{2}F(x) \parallel
\leq \alpha$ para todo $x \in S$ e $\sigma = [v_{0}, \ldots, v_{n}]$
um simplexo de uma triangula\c c\~ao robusta $T$ de $S$ $(\theta(T) > 0)$, ent\~ao 
\begin{enumerate}
\item $\parallel F(v) - F_{T}(v) \parallel \leq \alpha \rho^{2}(T)/2$ 
para todo $v \in S$ e 
\item $\parallel DF(v) - DF_{T}(v) \parallel \leq \alpha \rho(T)/\theta(T)$ para 
todo $v$ no interior de simplexos de dimens\~ao $n$ de $T$.
\end{enumerate}
\end{teore}

Embora $DF_{\sigma}(v)$ esteja bem definida para todo simplexo
$\sigma$ de dimens\~ao $n$, se $\tau$ \'e uma face comum de dois
simplexos $\sigma_{1}$ e $\sigma_{2}$ de  dimens\~ao $n$ e $v \in
\tau, DF_{\sigma_{1}}(v)$ e $DF_{\sigma_{2}}(v)$ podem ser distintas
e, portanto, $DF_{T}(v)$ n\~ao est\'a bem definida.



%\markboth{Aproxima\c{c}\~oes de Fun\c{c}\~oes Impl\'icitas}{Esquema de Diferen\c{c}as Simplicial}
%\section{Esquema de Diferen\c{c}as Simplicial}\label{vi_eds}
%
%Esta se��o e a pr�xima s�o importante para mostrar o funcionamento das aproxima��es de variedades impl�citas e suas limita��es.
%
%Seja $F : S \subset  \R^{n} \rightarrow \R^{k}$ uma 
%aplica\c c\~ao de classe $C^{2}$ e $T$ uma triangula\c c\~ao de $S$.
%
%Seja $\sigma = [v_{0}, \ldots, v_{n}] \in T$. Como $\sigma$ \'e um
%simplexo de dimens\~ao $n$, os vetores 
%$\{ v_{1}-v_{0}$, $v_{2}-v_{0},\ldots,v_{n}-v_{0}\}$ s\~ao linearmente 
%independentes; logo, fazendo \\ $\delta_{i} = \parallel v_{i}-v_{0} \parallel$ e 
%$\omega_{i} = (v_{i}-v_{0})/\delta_{i}$, $i = 1,\ldots,n$, temos que a
%matriz cujas colunas s\~ao os vetores $\{\omega_{1},\ldots,\omega_{n}\}$ \'e 
%invert\'{\i}vel. 
%
%Para $v \in \sigma$, seja $\lambda$, tal que $\sum_{i=0}^{n} \lambda_{i} v_{i} 
%= v$, $\sum_{i=0}^{n} \lambda_{i} = 1$ e $\lambda_{i} \geq 0$, 
%$i = 0,\ldots,n$. Ent\~ao, pode-se escrever 
%$$
%\begin{array}{l}
  %v - v_{0} = \\
 %\sum^{n}_{i=0} \lambda_{i} v_{i} - v_{0} \sum^{n}_{i=0} \lambda_{i} = \\
 %\sum^{n}_{i=1} \lambda_{i} (v_{i} - v_{0}) = \\
 %\sum^{n}_{i=1} \lambda_{i} \delta_{i} \omega_{i} = \\
 %\left( \omega_{1} \cdots \omega_{n} \right)
  %\left(  
  %\begin{array}{c}
  %\lambda_{1} \delta_{1} \\
  %\vdots \\
  %\lambda_{n} \delta_{n} 
  %\end{array}
  %\right)
%\end{array}
%$$
%ou
%$$ 
%\left( 
%\begin{array}{c}
%\lambda_{1} \delta_{1} \\
%\vdots \\
%\lambda_{n} \delta_{n}
%\end{array}
%\right) 
 %= 
%\left( \omega_{1} \cdots \omega_{n} \right)^{-1} 
%\left( v - v_{0} \right). 
%$$
%
%Agora
%$$
%\begin{array}{l}
  %F_{T}(v) - F(v_{0}) = \\
 %F_{\sigma}(v) - F(v_{0}) = \\
 %\sum^{n}_{i=0} \lambda_{i} F(v_{i}) - F(v_{0}) \sum^{n}_{i=0} \lambda_{i} =\\
 %\sum^{n}_{i=1} \lambda_{i} (F(v_{i}) - F(v_{0})) = \\
 %\sum^{n}_{i=1} \lambda_{i} \delta_{i} (F(v_{i})- F(v_{0}))/\delta_{i}  
%\end{array}
%$$
%ou
%$$
%\begin{array}{l}
  %F_{T}(v) - F(v_{0}) = \\
 %\left( \frac{F(v_{1})-F(v_{0})}{\delta_{1}} \cdots 
         %\frac{F(v_{n})-F(v_{0})}{\delta_{n}} \right) 
  %\left(
  %\begin{array}{c}
  %\lambda_{1} \delta_{1} \\
  %\vdots \\
  %\lambda_{n} \delta_{n}
  %\end{array}
  %\right) = \\
 %\left( \frac{F(v_{1})-F(v_{0})}{\delta_{1}} \cdots 
         %\frac{F(v_{n})-F(v_{0})}{\delta_{n}} \right) 
  %\left( \omega_{1} \cdots \omega_{n} \right)^{-1} 
  %\left( v-v_{0} \right).
%\end{array}
%$$
%
%Da\'{\i}, como $\omega_{i} = (v_{i}-v_{0})/\delta_{i}$, tem-se
%$v_{i} = v_{0}+\delta_{i}\omega_{i}$, $i= 1,\ldots,n$, e portanto
%{\scriptsize
%$$
%F_{T}(v) - F(v_{0}) = 
%\left(
%\frac{F(v_{0}+\delta_{1}\omega_{1})-F(v_{0})}{\delta_{1}}
%\cdots 
%\frac{F(v_{0}+\delta_{n}\omega_{n})-F(v_{0})}{\delta_{n}}
%\right)
%\left(
%\omega_{1} \cdots \omega_{n}
%\right)^{-1}  
%\left(
%v-v_{0} 
%\right). 
%$$
%}
%
%Observe que as colunas da matriz
%$$
%\left(
%\frac{F(v_{0}+\delta_{1}\omega_{1})-F(v_{0})}{\delta_{1}}
%\cdots  
%\frac{F(v_{0}+\delta_{n}\omega_{n})-F(v_{0})}{\delta_{n}}
%\right)
%$$
%s\~ao aproxima\c c\~oes para as derivadas direcionais de
%$F$ nas dire\c c\~oes $\omega_{1},\ldots,\omega_{n}$, em torno do ponto
%$v_{0}$; portanto este \'e um esquema de diferen\c cas que 
%chamaremos de esquema de diferen\c cas simplicial. 
%
%Assim, se $\delta_{i} \rightarrow 0$ para $i= 1, \ldots, n$ com
%uma certa proporcionalidade, isto \'e, se o di\^ametro de $\sigma$
%tende a zero mantendo a robustez limitada para que os vetores 
%$\omega_{1},\ldots,\omega_{n}$ continuem sendo linearmente independentes, 
%ent\~ao $F_{\sigma}$ \'e uma apro\-xima\c c\~ao para os dois primeiros 
%termos da s\'erie de Taylor de $F$, em torno do ponto $v_{0}$.
%Outra observa\c c\~ao \'e que a mesma an\'alise pode ser feita para
%qualquer outro v\'ertice de $\sigma$; portanto, o esquema descrito anteriormente
%n\~ao depende apenas do ponto $v_{0}$ e sim do simplexo $\sigma$.
%

\markboth{Aproxima\c{c}\~oes de Fun\c{c}\~oes Impl\'icitas}{Resultados sobre Aproxima��es Simpliciais}
\section{Resultados sobre Aproxima��es Simpliciais}\label{vi_ari}


Seja $F : U \rightarrow \R^{k}$ uma aplica\c c\~ao de classe $C^{p}$ no aberto $U \subset \R^{n}$, com $n > k$, $p > max\{ 1, n/k -1\}$ e tendo ${\bf 0} \in \R^{k}$ como valor regular. Considere a variedade de dimens\~ao $n-k$ e classe $C^{p}$, $\mathcal{M} = F^{-1}({\bf 0})$ e uma vizinhan\c ca tubular de $\mathcal{M}$, $V_\mathcal{M}$, tal como no Teorema \ref{teo_VT}.

Como desejamos uma aproxima\c c\~ao de $\mathcal{M}$ que possa ser representada computacionalmente, vamos restringir o dom\'{\i}nio de $F$ a um compacto $S \subset U$. Assim, existe $\alpha > 0$ tal que $\parallel D^{2} F(x)\parallel \leq \alpha$ para todo $x \in K$ e, portanto, se $T$ for uma triangula\c c\~ao robusta de $S$, fica valendo o Teorema \ref{teo_AG} para $F_{T}$.

Para estabelecermos algumas propriedades sobre a aproxima\c c\~ao $\mathcal{M}_{T} = F_{T}^{-1}({\bf 0})$ de $\mathcal{M}$, vamos observar algumas propriedades que $F$ e $T$ devem satisfazer. 

Seja $\sigma = [v_{0}, \ldots, v_{k}]$ um simplexo de dimens\~ao $k$ de $U$. Para que $F_{\sigma}(\sigma) = [F(v_{0}),\ldots,F(v_{k})]$ seja um simplexo de dimens\~ao $k$ em $\R^{k}$, basta que os de vetores $F(v_{1})-F(v_{0}),\ldots,F(v_{k})-F(v_{0})$ sejam linearmente independentes, pelo fato de $F_{\sigma}$ ser uma aplica\c c\~ao afim.
Neste caso, se $\tau$ \'e uma face de $\sigma$ de dimens\~ao $r \leq k$, teremos que $F_{\sigma}(\tau)$ \'e uma face de $F_{\sigma}(\sigma)$ de dimens\~ao $r$. 

Com rela\c c\~ao a este assunto temos o seguinte teorema:

\begin{teore} {\rm $($Veja~\cite{Fr91}$)$} \label{teo_Fr91_1}
Para quase todo simplexo $\sigma$ de dimens\~ao $r \leq k$ de $S \cap V_\mathcal{M}$, temos que $F_{\sigma}(\sigma)$ \'e um simplexo de dimens\~ao $r$ em $\R^{k}$.
\end{teore}


\begin{defi} 
Diremos que $v \in \sigma = [v_{0},\ldots,v_{n}] \in T$ \'e um ponto regular de $F_{T}$, se $DF_{\sigma}(v)$ for sobrejetora, ou de forma equivalente, que a matriz 
$$
\left( 
\begin{array}{ccc}
1 & \cdots & 1 \\
F(v_{0}) & \cdots & F(v_{n})
\end{array} 
\right)
$$
tenha posto $k+1$. 
Se $v$ n\~ao for um ponto regular de $F_{T}$, diremos que
\'e um ponto cr\'{\i}tico de $F_{T}$. 
Diremos que $c \in \R^{k}$ \'e valor regular de $F_{T}$ se todo $v \in F_{T}^{-1}(c)$ for ponto regular de $F_{T}$. 
Se $c$ n\~ao for valor regular de $F_{T}$, diremos que \'e valor cr\'{\i}tico de $F_{T}$.
\end{defi} 

\begin{teore} {\rm $($Veja~\cite{Eaves76}$)$} \label{teo_Ea76}
Seja $T$ uma triangula��o de $S$. Se $0 \in \R^{n}$ for valor regular de $F_{T}$, ent\~ao $\mathcal{M}_{T} = F_{T}^{-1}(0)$ \'e uma variedade linear por partes de dimens\~ao $n-k$.
\end{teore} 

Observe-se que o teorema \ref{teo_Ea76} n\~ao garante que $F_{T}^{-1}(0) \cap \sigma$ seja uma c\'elula de dimens\~ao $r$ se $\sigma$ for um simplexo de dimens\~ao $k+r$, $r = 1,\ldots,n-k$.  
Com rela\c c\~ao a esta observa\c c\~ao, temos o seguinte exemplo:
$ F : \R^{2} \rightarrow \R$ definida por $F(x,y) = x + 2y$ e 
$\sigma = [v_{0},v_{1},v_{2}]$ com $v_{0} = (0, 0), v_{1} = (1, 0)$ e 
$v_{2} = (0, 1)$. Como $F(v_{0}) = 0, F(v_{1}) = 1$ e $F(v_{2}) = 2$, temos
que $F_{\sigma}^{-1}(0) \cap \sigma = \{(0, 0)\}$, que \'e uma
c\'elula de dimens\~ao $0$.

Como $DF(x)$ tem posto $k$ para todo $x \in S \cap V_\mathcal{M}$, a inversa de Moore-Penrose para $DF(x)$, $DF(x)^{+} = DF(x)^{t} \cdot (DF(x) \cdot DF(x)^{t})^{-1}$ est\'a bem definida. 
Como $S \cap V_\mathcal{M}$ \'e um conjunto compacto, existe $\mu > 0$, tal que $\parallel DF(x)^{+} \parallel \leq \mu$ para todo $x \in S \cap V_\mathcal{M}$. 

\begin{teore} {\rm $($Veja~\cite{Fr91}$)$} \label{teo_Fr91_2}
Para quase todo simplexo $\sigma$ de dimens\~ao $k$ de $S \cap V_\mathcal{M}$ com $\alpha \mu \rho(\sigma)/\theta(\sigma) < 1$, cujo interior intercepta $\mathcal{M}$, temos que o interior de $\sigma$ intercepta $\mathcal{M}_{\sigma}$. 
\end{teore} 

Como $S$ \'e compacto, para quase toda triangula\c c\~ao $T$ de $S$, seus simplexos de dimens\~ao maior ou igual a $k$ t\^em interior transversal a $\mathcal{M}$. 
Da\'{\i}, segue o corol\'ario do teorema \ref{teo_Fr91_2}.

\begin{corol} {\rm $($Veja~\cite{Fr91}$)$} \label{cor_Fr91}
Para quase toda triangula\c c\~ao robusta $T$ de $S$ com di\^ametro suficientemente pequeno, temos que todo simplexo de dimens\~ao maior ou igual a $k$ tem interior transversal a $\mathcal{M}$ e $\mathcal{M}_{T}$, e se intercepta $\mathcal{M}$, ent\~ao tamb\'em intercepta $\mathcal{M}_{T}$. 
\end{corol} 
 
O corol\'ario \ref{cor_Fr91} nos d\'a duas propriedades muito importantes: 
para quase toda triangula\c c\~ao $T$ de $S$ com di\^ametro suficientemente
pequeno, temos 
\begin{itemize}
\item se $\sigma$ for um simplexo de dimens\~ao $k$ de $T$ e $x \in \sigma \cap \mathcal{M}$, existe $v \in \sigma \cap \mathcal{M}_{T}$ e consequentemente $d(x, v) = \parallel x-v \parallel \leq \rho(\sigma) \leq \rho(T)$; 
\item se $\sigma$ for um simplexo de dimens\~ao $k+r$ de $T$ com $0 \leq r \leq n-k$, temos que se $F_{T}^{-1}(0) \cap \sigma \neq \emptyset$, ent\~ao $F_{T}^{-1}(0) \cap \sigma$ \'e uma c\'elula convexa afim de dimens\~ao $r$. 
\end{itemize}

A segunda propriedade ser\'a muito importante para representa\c c\~ao computacional das c\'elulas de $\mathcal{M}_{T}$. 
Tamb\'em com rela\c c\~ao \`a aproxima\c c\~ao entre $\mathcal{M}$ e $\mathcal{M}_{T}$, temos os seguintes teoremas:

\begin{teore} {\rm $($Veja~\cite{GaZa80}$)$} \label{teo_GaZa80}
Dado $\epsilon > 0$, existe uma triangula\c c\~ao $T$ de $S$ tal que, se $v \in \mathcal{M}_{T} = F_{T}^{-1}(0)$, ent\~ao $d(v,\mathcal{M}) = inf \{\parallel u-v\parallel ; u \in \mathcal{M} \} < \epsilon$. 
\end{teore} 

\begin{teore} {\rm $($Veja~\cite{AlGe80}$)$} \label{teo_AG80}
Sejam $x \in \mathcal{M}$ e $y \in \mathcal{M}_{T}$. 
Se $\parallel x-y \parallel \leq 1/(\alpha \mu)$, ent\~ao $\parallel x-y \parallel \leq \alpha \mu \rho^{2}(T)$.
\end{teore} 


