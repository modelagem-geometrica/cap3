%\documentclass{beamer}[10]
\documentclass[serif,10pt]{beamer}
\usetheme{Madrid}
\usepackage[latin1]{inputenc}
\usepackage[brazil]{babel}
\usepackage{multimedia}
\usepackage{pgfgantt}
\usepackage{overpic}
\usepackage{psfrag}
\usepackage{booktabs}
\usepackage{xcolor}
\usepackage{graphicx,xr}
\usepackage[T1]{fontenc}		% usa fontes postscript com acentos
\setbeamersize{text margin left=20pt,text margin right=20pt} % para
%\input{packages} %aqruivo latex onde est�o salvados os pacotes.
%\usepackage[pdftex]{hyperref} % link
\usefonttheme{professionalfonts}
\usepackage{psfrag}
\usepackage{eqnarray,amsmath}
\usepackage{verbatim}

\renewcommand{\theequation}{\thesection.\arabic{equation}}
\newtheorem{lema}{Lema}
\newtheorem{teore}{Teorema}
\newtheorem{defi}{Defini��o}
\newtheorem{corol}{Corol�rio}
\newtheorem{ex}{Exemplo}
\newtheorem{Codigo}{C�digo em Octave}
\newtheorem{propri}{Propriedade}
\newcommand{\dem}{\noindent \underline{Demonstra��o}: $\,$}
\newcommand{\fim}{\hfill $\rule{2.0mm}{2.0mm}$ \\}    
\newcommand{\R}{\mathbb{R}}
\newcommand{\Z}{\mathbb{Z}}
\newcommand{\C}{\mathbb{C}}
\renewcommand{\sectionmark}[1]{\markright{\thesection\ #1}}

%\DeclareCaptionType{Codigo}

\setbeamertemplate{caption}[numbered]

\usetheme{Madrid}
\usecolortheme{seahorse}
%%%%%%%%%%%%%%%%%%%%%%%%%%%%%%%%%%%%%%%%%%

\title[]{{\bf \huge{Variedades Computacionais}}}
   \vspace{3.5cm}
 \author[Variedades Computacionais]{
         {\huge \color{blue} Cap�tulo 4} \\ \vspace{0.5cm}
         {\large Antonio Castelo e Juliana Bertoco} \vspace{0.5cm}} 
 \date{ }

 
%%%%%%%%%%%%%%%%%%%%%%%%%%%%%%%%%%%%%%%%%%%%%%%%%%%%%%%%%%%%%%%%%%%%%%%%%%%%%%%%%%%
%%%%%%%%%%%%%%%%%%%%%%%%%%%%%%%%%%%%%%%%%%%%%%%%%%%%%%%%%%%%%%%%%%%%%%%%%%%%%%%%%%%
\begin{document}

\begin{frame}
\maketitle
\vspace{-2.0cm}
\begin{figure}[] 
\begin{center} 
   \includegraphics[clip,scale=0.4]{imagens/Olga_Sorkine_3.png} 
\end{center}
\end{figure} 
\end{frame} 
 
%%%%%%%%%%%%%%%%%%%%%%%%%%%%%%%%%%%%%%%%%%%%%%%%%%%%%%%%%%%%%%%%%%%%%%%%%%%%%%%%%%%

\pgfdeclareimage[height=1.8cm]{logo}{imagens/olga.png} 
\logo{\pgfuseimage{logo}}

%%%%%%%%%%%%%%%%%%%%%%%%%%%%% SUMARIO %%%%%%%%%%%%%%%%%%%%%%%%%%%%%%%%%%%%%%

\begin{frame}[allowframebreaks]
\frametitle{Sum�rio}
\tableofcontents
\end{frame}

%%%%%%%%%%%%%%%%%%%%%%%%%%%%%%%%%%%%%%%%%%%%%%%%%%%%%%%%%%%%%%%%%%%%%%%%%%%%%%%%%%%
\centering

%%%%%%%%%%%%%%%%%%%%%%%%%%%%%%%%%%%%%%%%%%%%%%%%%%%%%%%%%%%%%%%%%%%%%%%%%%%%%%%%%%%
%%%%%%%%%%%%%%%%%%%%%%%%%%%%%%%%%%%%%%%%%%%%%%%%%%%%%%%%%%%%%%%%%%%%%%%%%%%%%%%%%%%
\section{Cap�tulo 4:  Modelagem Utilizando Operadores Diferenciais}

%%%%%%%%%%%%%%%%%%%%%%%%%%%%% SLIDE 4.0 %%%%%%%%%%%%%%%%%%%%%%%%%%%%%%%%%%%%%%%%%%%
\begin{frame}{{\bf \color{blue} Cap�tulo 4}}

\begin{block}{\bf Cap�tulo 4: Modelagem Utilizando Operadores Diferenciais}
Neste cap�tulo � apresentado um m�todo de reconstru��o de superf�cies lineares por partes utilizando o operador de Laplace-Beltrami discreto.
\end{block}

\end{frame}


%%%%%%%%%%%%%%%%%%%%%%%%%%%%%%%%%%%%%%%%%%%%%%%%%%%%%%%%%%%%%%%%%%%%%%%%%%%%%%%%%%%
\subsection{Operador de Laplace-Beltrami}
%%%%%%%%%%%%%%%%%%%%%%%%%%%%%%%%%%%%%%%%%%%%%%%%%%%%%%%%%%%%%%%%%%%%%%%%%%%%%%%%%%%

%%%%%%%%%%%%%%%%%%%%%%%%%%%%% SLIDE 4.1 %%%%%%%%%%%%%%%%%%%%%%%%%%%%%%%%%%%%%%%%%%%
\begin{frame}{{\bf \color{blue} Operador de Laplace-Beltrami}}

\begin{block}{\bf Operador de Laplace-Beltrami}
A equa��o de Laplace-Beltrami � uma varia��o da equa��o de Laplace que tem propriedades geom�tricas importantes das superf�cies que s�o solu��o desta equa��o.
\end{block}

\begin{figure}[h]
\begin{center} 
{\includegraphics[angle=0,scale=0.55]{imagens/Olga_Sorkine_2}}
\caption{Superf�cie deform�vel.} 
\label{fig.Deformavel}
\end{center}
\end{figure}

\end{frame}

%%%%%%%%%%%%%%%%%%%%%%%%%%%%%%%%%%%%%%%%%%%%%%%%%%%%%%%%%%%%%%%%%%%%%%%%%%%%%%%%%%%
\subsection{Reconstru��o de Variedades Lineares por Partes}
%%%%%%%%%%%%%%%%%%%%%%%%%%%%%%%%%%%%%%%%%%%%%%%%%%%%%%%%%%%%%%%%%%%%%%%%%%%%%%%%%%%

%%%%%%%%%%%%%%%%%%%%%%%%%%%%% SLIDE 4.2 %%%%%%%%%%%%%%%%%%%%%%%%%%%%%%%%%%%%%%%%%%%
\begin{frame}{{\bf \color{blue} Reconstru��o de Variedades Lineares por Partes}}

\begin{block}{\bf Reconstru��o de Variedades Lineares por Partes}
O operador de Laplace-Beltrami discreto pode ser utilizado para a reconstru��o de variedades de dimens�o 1, 2 ou 3, lineares por pates a partir de poucas amostras na variedade original.
\end{block}

\begin{figure}[h]
\begin{center} 
{\includegraphics[angle=0,scale=0.75]{imagens/Olga_Sorkine}}
\caption{Superf�cie deform�vel.} 
\label{fig.Deformavel2}
\end{center}
\end{figure}

\end{frame}

%%%%%%%%%%%%%%%%%%%%%%%%%%%%%%%%%%%%%%%%%%%%%%%%%%%%%%%%%%%%%%%%%%%%%%%%%%%%%%%%%%%
%%%%%%%%%%%%%%%%%%%%%%%%%%%%%%%%%%%%%%%%%%%%%%%%%%%%%%%%%%%%%%%%%%%%%%%%%%%%%%%%%%%
\section{Equipe}

%%%%%%%%%%%%%%%%%%%%%%%%%%%%% Equipes %%%%%%%%%%%%%%%%%%%%%%%%%%%%%%%%%%%%%%%%%%%
\begin{frame}{{\bf \color{blue} Equipe}}

\begin{block}{\bf Equipe}
\begin{itemize}
\item {\bf Cap�tulo 4:} Pedro e Fakhoury;
\end{itemize}
\end{block}

\end{frame}

%%%%%%%%%%%%%%%%%%%%%%%%%%%%%%%%%%%%%%%%%%%%%%%%%%%%%%%%%%%%%%%%%%%%%%%%%%%%%%%%%%%
%%%%%%%%%%%%%%%%%%%%%%%%%%%%%%%%%%%%%%%%%%%%%%%%%%%%%%%%%%%%%%%%%%%%%%%%%%%%%%%%%%%
\section{Refer�ncias}

%%%%%%%%%%%%%%%%%%%%%%%%%%%%% Refer�ncias %%%%%%%%%%%%%%%%%%%%%%%%%%%%%%%%%%%%%%%%%%%
\begin{frame}{{\bf \color{blue} Refer�ncias}}

\begin{block}{\bf Refer�ncias}
\begin{itemize}
\item Artigos da Olga Sorkine;
\item \url{https://igl.ethz.ch/publications/}
\item \url{https://igl.ethz.ch/code/}
\item \url{https://igl.ethz.ch/projects/Laplacian-mesh-processing/Laplacian-mesh-editing/index.php}
\end{itemize}
\end{block}

\end{frame}

%%%%%%%%%%%%%%%%%%%%%%%%%%%%%%%%%%%%%%%%%%%%%%%%%%%%%%%%%%%%%%%%%%%%%%%%%%%%%%%%%%%
%%%%%%%%%%%%%%%%%%%%%%%%%%%%%%%%%%%%%%%%%%%%%%%%%%%%%%%%%%%%%%%%%%%%%%%%%%%%%%%%%%%

\end{document} 

 
